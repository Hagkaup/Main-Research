\documentclass[11pt]{article}

\title{\textbf{Notes on adaptive radiation and dispersal}}
\author{Allan Edelsparre}
\date{}

\begin{document}

\maketitle

\section{Adaptive radiation}

Key to adaptive radiation is the notion of a monophyletic group with ancestral influence on adaptive traits. For example, the ancestral state should be flexible in terms of trais and that may open up for new niches or resources to be exploited (e.g. flexible stem model).

There are binary radiations. Effectively, species pairs that repeatedly arise in similar environments ``replicate radiation''. This ancestral trait variation (flexibility) may be a behavioural predisposition as is proposed in examples of resource polymorphisms. Wilson and McLaughlin (2007) talks about the importance of looking at behavioural trait variation in charr systems where morphlogy is absent. This would work well with the Smith and Skulason (1996) model of evolution via resource polymorphism. There are charr examples as well as the classic three-spined stickleback species pair example. I have also added a paper that talks about anitipredator behaviour and migration in Atlantic salmon by felicity Huntingford. This paper may put a perspective on linking invasion with foraging behaviour/antipredator behaviour. 

\end{document}
