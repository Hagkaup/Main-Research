Git commands and other details

User name: Hagkaup
Password: 11Wineglass


If it’s your first push to a repo then you should use that command to tell it what branch to start saving tracking to:

git push -u origin master


Adding a file:

git add <filename> <another file> <third file>

Or, adding all files:

git add *

Then commit my changes:

git commit -m "these are the git details to familiarize myself with the system"

Save the changes and push:

git push


# navigating through terminal or eshell:

cd ..

# the above command move up one directory

cd ~/

# the above command will move to my home folder

mkdir

# the above command makes a directory.

## How to open files using the default programs

xdg-open Git_Details.tex ## for opening a tex document

xdg-open Desktop ## for opening a folder




### February 26 Bioclub

grep -w 'this' file.text | wc -c

This passes a line that will count words, but the -c command will make sure it uses the character option rather than just the word count.

emacs Git_Details.tex


### For loops and other syntax

 if [argument]; then
 [function]

### Same with else and so on, following the R logic

### for loops are slow. Same syntax as R:

for number in {1..5}
do
echo $number
done

### read 'combining for loops with conditionals
