Git commands and other details


## To create a pdf from a tex file use the following in command line:

pdflatex <filename.tex>



If it’s your first push to a repo then you should use that command to tell it what branch to start saving tracking to:

git push -u origin master


Adding a file:

git add <filename> <another file> <third file>

Or, adding all files:

git add *

Then commit my changes:

git commit -m "these are the git details to familiarize myself with the system"

Save the changes and push:

git push

### Other codes to help with commanding git language:

# Switching branches. One of the issues is that once you are in a specific branch on your local drive the files from the other branches in that change to the other branch, so you won't find them on your local drive unless you switch branch.

git stash (this saves changes and cleans up)

git branch -a (this lists all the branches)

git checkout branch master (switched to the master branch)




# navigating through terminal or eshell:

cd ..

# the above command move up one directory

cd ~/

# the above command will move to my home folder

mkdir

# the above command makes a directory.

## How to open files using the default programs

xdg-open Git_Details.tex ## for opening a tex document

xdg-open Desktop ## for opening a folder




### February 26 Bioclub

grep -w 'this' file.text | wc -c

This passes a line that will count words, but the -c command will make sure it uses the character option rather than just the word count. I can do this with single characters or series of letters such as stop codons for example 'AUG' or some other combination of letters. I tried this one out in terminal and it works really well.






emacs Git_Details.tex


### For loops and other syntax

 if [argument]; then
 [function]

### Same with else and so on, following the R logic

### for loops are slow. Same syntax as R:

for number in {1..5}
do
echo $number
done

### read 'combining for loops with conditionals
